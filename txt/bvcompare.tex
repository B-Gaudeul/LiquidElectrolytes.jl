\RequirePackage[l2tabu, orthodox,abort]{nag}

\InputIfFileExists{jfsettings}{}{
\documentclass[12pt,oneside,reqno]{amsart}
\usepackage[a4paper,portrait,left=1.5cm,outer=3.5cm,headheight=15pt,bottom=3cm,top=3cm]{geometry} 
}

%%%%%%%%%%%%%%%%%%%%%%%%%%%%%%%%%%%%%%%%%%%%%%%%%%%%%%%%%%%%%%%&
% Draft settings
\usepackage{xcolor}
\usepackage[color]{showkeys}
\definecolor{refkey}{rgb}{0.3,0.5,0.3}
\definecolor{labelkey}{rgb}{0.3,0.5,0.3}
\definecolor{todo}{rgb}{0.6,0.9,0.5}
\usepackage[textwidth=3cm,textsize=tiny,color=todo]{todonotes}


%%%%%%%%%%%%%%%%%%%%%%%%%%%%%%%%%%%%%%%%%%%%%%%%%%%%%%%%%%%%%%%%%
%%  GENERAL SETTINGS

\usepackage{amsmath,amssymb,amsfonts,amsthm} % ams stuff everywhere
\usepackage{array}                 % math in tabular etc.
\usepackage{booktabs}              % improve table rendering
\usepackage[utf8]{inputenc}        % UTF8 fonts
\usepackage[T1]{fontenc}           % Fix font encoding issues with accented chars
\usepackage[english]{babel}        % Language awareness
\usepackage{csquotes}              % Context sensitive qouting
\usepackage{enumitem}              % no dull items
\usepackage{tikz}
\usepackage{verbatim}
\usepackage{graphicx}
\usepackage{fouriernc}
\usepackage[amssymb,thickspace,thinqspace]{SIunits}

\graphicspath{ {./} {./pic/}}
\numberwithin{equation}{section}




\title[Verification of Butler-Volmer derivation]{Numerical verification of the derivation of the Butler-Volmer equation from
the Nernst-Planck equations\\ Draft}
\author{$\dots$}
\date{\today}



\begin{document}
\maketitle

\section{Introduction}

The authors of \cite{dreyer2016new} provides a thermodynamically and  mathematically motivated  derivation of
Butler-Volmer - like  equations assuming that 
\begin{itemize}
\item all reactions take place at the interface between the electrolyte
and the metal (interface I)
\item in a boundary layer defined by the plane of zero charge, short relaxation
  times allow the assumption that the electrochemical potential of the dissolved species
  is constant
\end{itemize}
A similar analysis already was performed in \cite{rubinstein1990electro}, albeit without the
connection to the Butler-Volmer equation.

This paper attempts to provide  an assessment of this derivation using
numerical    computations   based    on   Nernst-Planck-Poisson    and
Nernst-Planck-Ohm systems.

%%%%%%%%%%%%%%%%%%%%%%%%%%%%%%%%%%%%%%%%%%%%%%%%%%%%%%%%%%%%
%%%%%%%%%%%%%%%%%%%%%%%%%%%%%%%%%%%%%%%%%%%%%%%%%%%%%%%%%%%%
\section{Electrolyte models}

In this  section, we  introduce three  electrolyte models  and discuss
their boundary conditions and their interconnections. For the notations,
see table \ref{tab:notations}.

%%%%%%%%%%%%%%%%%%%%%%%%%%%%%%%%%%%%%%%%%%%%%%%%%%%%%%%%%%%%
\subsection{Nernst-Planck-Poisson  with  volume constraints} 

\begin{subequations}\label{sys:PNP}
\begin{align}
  -\nabla \varepsilon \nabla \phi&= q\\
  \partial c_i  + \nabla \cdot \mathbf N_i  &=0 & (i=1\dots N)\\
  \mathbf N_i &= - \frac{D_i}{RT} c_i \left( \nabla \mu_i - \frac{M_i}{M_0}\nabla \mu_0 + z_i F \nabla \phi \right)& (i=1\dots N) \label{eq:NP}
\end{align}
where $q=F\sum_{i=1}^N z_i c_i$.
From the momentum balance at zero barycentric velocity, the pressure $p$ obeys the relationship $\nabla p = q \nabla \phi$
which can be replaced \cite{Fuhrmann2015} by :
\begin{align}
  -\Delta p + \nabla \cdot q \nabla \phi \label{eq:press}
\end{align}
The system is closed by the incompressibility condition
\begin{align}\label{eq:incompress}
  \sum_{i=0}^N \bar v_ic_i&=1
\end{align}
and the relationship between chemical potentials and concentrations \cite{dreyer2013overcoming} :
\begin{align}\label{eq:constrel}
  \mu_i &= \mu_i^\circ + \bar v_i(p-p^\circ) + RT \ln \frac{c_i}{\bar c}  & (i=0\dots N)
\end{align}
where $  \bar c = \sum_{i=0}^N  c_i$.
\end{subequations}

By \eqref{eq:incompress} one can obtain the solvent concentration $c_0$  from the solute concentrations
$c_1\dots c_N$:
\begin{align}
  \label{eq:c0}
  c_0&=\frac{1}{v_0} -  \sum_{i=1}^N  \frac{\bar v_i}{v_0}c_i
\end{align}
The diffusive mass fluxes are defined relative to the barycentric velocity \cite{dreyer2013overcoming,Fuhrmann2015}
\begin{align}
  \sum_{i=0}^N M_i\mathbf N_i&=0
\end{align}




For the derivation of the numerical scheme we calculate the non-constant
contribution to the excess chemical potential defined as the difference between the chemical potential
contribution to the driving force $\mu_i - \frac{M_i}{M_0}\mu_0$ and the chemical potential of an ideal gas
$RT \log \frac{c_i}{\bar c}$:
For $i=1\dots N$ we have
\begin{align}
  \mu_i^{ex} &=  \mu_i - \frac{M_i}{M_0} \mu_0  - RT \log \frac{c_i}{\bar c} \nonumber\\
             &= \bar v_ip +  RT \log \frac{c_i}{\bar c}  -\frac{M_i}{M_0}\left(  v_0p + RT \log \frac{c_0}{\bar c}\right)  - RT \log \frac{c_i}{\bar c} \nonumber\\
            &= \bar v_ip  -\frac{M_i}{M_0}\left(  v_0p + RT \log \frac{c_0}{\bar c}\right)\nonumber\\
            &= \left(\bar v_i-\frac{M_i}{M_0}v_0\right)p - \frac{M_i}{M_0}RT\log \frac{c_0}{\bar c} \label{eq:muex}
\end{align}
and express $c_0$ via \eqref{eq:c0} and $\bar c$ in terms of $c_1\dots c_N$
\begin{align*}
  \bar c &% = \frac{1}{v_0} -  \sum_{i=1}^N \frac{\bar v_i}{v_0}c_i + \sum_{i=1}^N  c_i
          = \frac{1}{v_0} + \sum_{i=1}^N \left(1- \frac{\bar v_i}{v_0}\right) c_i 
\end{align*}

Thus, we can rewrite \eqref{eq:NP}  as 
\begin{align}\label{eq:flux}
  \mathbf N_i &= - D_i\bar c_i \left(\nabla \frac{c_i}{\bar c_i} + \frac{c_i}{\bar c_iRT} (\mu_i^{ex} + z_i F \nabla \phi) \right)
\end{align}
where the expression \eqref{eq:muex} for $\mu_i^{ex}(c_1\dots c_N, p)$  encodes the non-ideality properties of the mixture.
The excess chemical potential is linked to the \textit{activity coefficient} $\gamma_i=\exp(\frac{\mu_i}{RT})$ and the
reciprocal (inverse) activity coefficient $\beta_i=\frac1{\gamma_i}$  \cite{Fuhrmann2015}.
These data allow to define various numerical flux approximations in the context of the Voronoi finite volume
method \cite{Fuhrmann2015,GaudeulFuhrmannNM2022}.

\subsection{Equilibrium problem}
\eqref{eq:flux} gives
\begin{align}\label{eq:flux}
  \mathbf N_i &= - D_i\frac{c_i}{RT}\nabla\left( \mu_i +  \mu_i^{ex} + z_i F \nabla \phi \right)
\end{align}
Zero flux gives 
\begin{align}
  \mu_i   &=   - \mu_i^{ex} - z_i F \nabla \phi\\
  c_i   &=  C_i\exp\left( -\frac{\mu_i^{ex} + z_i F \nabla \phi}{RT}\right)\\
        &=  C_i \exp\left(-\frac{\left(\bar v_i-\frac{M_i}{M_0}v_0\right)p - \frac{M_i}{M_0}RT\log \frac{c_0}{\bar c} + z_i F \nabla \phi}{RT}\right)
        &=  C_i \exp\left(-\frac{\left(\bar v_i-\frac{M_i}{M_0}v_0\right)p - \frac{M_i}{M_0}RT\log \frac{c_0}{\bar c} + z_i F \nabla \phi}{RT}\right)
\end{align}


\begin{table}
  \begin{tabular}{rl}
    $p$ & pressure\\
    $\phi $ &electrostatic potential\\
    $\mu_i$& species chemical potential\\
    $c_i$& species molar concentration\\
    $N_i$& species molar flux\\
    $v_i$ & species molar volume\\ 
    $M_i$ & species molar mass\\
    $\kappa_i$ & species solvation number\\
    $\bar v_i = v_i + \kappa_i v_0$ & solvated molar volume\\
    $z_i$ & species charge number\\
    $\varepsilon$& dielectric constant\\
    $F$ & Faraday constant\\
    $R$ & Molar gas constant\\
    $T$ & Temperature
  \end{tabular}
  \medskip
  
  \caption{Notations}
  \label{tab:notations}
\end{table}

%%%%%%%%%%%%%%%%%%%%%%%%%%%%%%%%%%%%%%%%%%%%%%%%%%%%%%%%%%%%%%%%%%%%%%%%%%%%%%%%%%%%%%%%%%%%%%%%
\subsection{Domain, boundary an initial conditions} 
\subsubsection{Full cell}
Assume a one-dimensional domain  $\Omega=(0,L)$, filled with
the   electrolyte,   and   with   electrodes   of   similar   reactive
characteristics  at  $x=0$  and   $x=L$.  Applying  an  electrostatic
potential difference $U=U_{0}-U_{L}$ to the system leads to Dirichlet boundary conditions
for the electrostatic potential:
\begin{subequations}\label{sys:PNPbc}
\begin{align}
  \phi|_{x=0}=U_{0}, &\quad \phi|_{x=L}=U_{L}
\end{align}
For the constituents of the electrolyte, we assume surface reactions which do not
involve the metal:
\begin{align}
  (N_i\cdot \mathbf n)|_{x=0}&=R_i(\mu_0^0\dots\mu_n^0), & (i=1\dots N)\\
  (N_i\cdot \mathbf n)|_{x=L} &=R_i(\mu_0^L\dots\mu_n^L) & (i=1\dots N).
\end{align}
Particular expressions will be discussed later. In addition to the chemical potentials
of the dissolved species, $R_i$ depend on the electron concentration in the metal
which is assumed to be constant and thus can be hidden in the constants describing $R_i$.
Assuming that surface and bulk chemical potentials are equal leads to the equivalences
\begin{align*}
  \mu_i^0=\mu_i(0), &\quad \mu_i^L=\mu_i(L) & (i=0\dots N).
\end{align*}

In order to fix the integration constant of the pressure, we set
\begin{align}
  p(L/2)=0
\end{align}
\end{subequations}

As initial conditions, we set  $c_i|_{t=0}=c_i^{bulk}$ ($i=1\dots N$) such that it fulfills
local electroneutrality $\sum_{i=1}^N z_ic_i^{bulk}=0$ and $c_0^{bulk}>0$.

\subsubsection{Half cell}
Here, we set bulk boundary conditions at $L=0$:
\begin{subequations}
  \begin{align}
    \phi|_{x=L}&=0\\
    c_i|_{x=L}&= c_i^{bulk}\\
    p|_{x=L}&=0
  \end{align}
\end{subequations}

%%%%%%%%%%%%%%%%%%%%%%%%%%%%%%%%%%%%%%%%%%%%%%%%%%%%%%%%%%%%%%%%%%%%%%%%%%%%%%%%%%%%%%%%%%%%%%%%
\subsection{Nernst-Planck-Poisson   system    with   electroneutrality constraint} 
For  large  domains, the  width  of  the polarization  boundary  layer
inherent to  the NPP-System is  negligible, it makes sense  to replace
the Poisson equation with the local electroneutrality constraint:
\begin{align}
  0&= \sum_{i=1}^N z_i c_i \label{eq:eneu}
\end{align}
This condition can be enforced just by setting  $\varepsilon=0$
As we replaced the Poisson equation by an algebraic condition, we need to remove 
the Dirichlet boundary conditions for the electrostatic potential from the system.

But we still keep the reaction boundary conditions:
\begin{subequations}\label{sys:NNPbc}
\begin{align}\
  (N_i\cdot \mathbf n)|_{x=0}&=R_i(\mu_0^0\dots\mu_n^0), & (i=1\dots N)\\
  (N_i\cdot \mathbf n)|_{x=L} &=R_i(\mu_0^L\dots\mu_n^L) & (i=1\dots N).
\end{align}
\end{subequations}

In \cite{guhlke2015theorie}, based on an approach introduced in \cite{caginalp1988dynamics}, the method
of matching asymptotic expansions has been used in order to derive boundary conditions for
this case. As a main conclusion, the assumption of local equilibrium 
in the polarization boundary layer is valid and 
the electrochemical potentials $\mu_i + z_i F \phi$ are constant in this boundary layer. Assuming that the electrostatic
potential at the metal side of the electrode at $x=0$ still assumes the applied potential value $U_0$, the
potential difference over the polarization boundary layer can be expressed as $\phi(0)-U_0$. A
similar consideration is true for $x=L$.
Now one can 
calculate the chemical potentials at the surface entering \eqref{sys:NNPbc} as
\begin{subequations}\label{sys:neubc}
   \begin{align}
     \mu_i^0=\mu_i(0) + z_i F(\phi(0)-U_0)\\
     \mu_i^L=\mu_i(L) + z_i F(\phi(L)-U_L)
   \end{align}
 \end{subequations}

 The pressure equation \eqref{eq:press} has a zero right hand side leading to a
 constant value of the pressure with the boundary conditions involved.

%%%%%%%%%%%%%%%%%%%%%%%%%%%%%%%%%%%%%%%%%%%%%%%%%%%%%%%%%%%%%%%%%%%%%%%%%%%%%%%%%%%%%%%%%%%%%%%%
\subsection{Nernst-Planck-Ohm system}

We assume that species number $N$ belongs to the background electrolyte which
is inert to  the reactiom, such that $R_N=0$. Further, we assume that 
the dilute solution assumption is valid, in particular, $c_0\approx\bar c\approx \mathrm{const}$.
With constant pressure, this leads to $\mu_i^{ex}=0$ resulting in 
the classical drift-diffusion expression
\begin{align}
  \mathbf N_i &= -D_i \nabla c_i  - z_i \frac{F}{RT} D_i c_i \nabla \phi \label{eq:ddflux} & (i=1\dots N)
\end{align}

The algebraic condition \ref{eq:eneu} allows to express the species density $c_{N}$ 
via the other densities:
\begin{align}
  c_N=-\sum_{i=1}^{N-1} \frac{z_i}{z_N} c_i
\end{align}
Furthermore, summing up all dissolved species flux equations weighted by $Fz_i$ leads to Ohm's law
coupled to the transport equations for the species $i=1\dots N-1$. Under the condition that all
diffusion coefficients are equal, $D_i=D$, one arrives at
\begin{subequations}\label{sys:ONP}
\begin{align}
  -\nabla\cdot (\sigma\nabla\phi) &= 0\\
  \partial c_i  + \nabla \cdot \mathbf N_i  &=0 & (i=1\dots N-1)\\
  \mathbf N_i &= -D \nabla c_i  - z_i \frac{F}{RT}  D c_i \nabla \phi \label{eq:xddflux} & (i=1\dots N-1)
\end{align}
\end{subequations}
with the boundary conditions
\begin{subequations}\label{ONPbc}
\begin{align}
  -\sigma\nabla\phi \cdot \mathbf n|_{x=0} &= \sum_{i=2}^N Fz_iR_i(\mu_0^0\dots\mu_{N-1}^0)\\
  (N_i\cdot \mathbf n)|_{x=0}&=R_i(\mu_0^0\dots\mu_{N-1}^0), & (i=1\dots N-1)\\
  -\sigma\nabla\phi \cdot \mathbf n|_{x=L} &= \sum_{i=1}^{N-1} Fz_iR_i(\mu_0^L\dots\mu_{N-1}^L)\\
  (N_i\cdot \mathbf n)|_{x=L} &=R_i(\mu_0^L\dots\mu_{N-1}^L) & (i=1\dots N-1)
\end{align}
\end{subequations}
The surface species concentration is calculated via \eqref{eq:constrel} (with constant $\bar c$) and
  \eqref{sys:neubc}.
The  conductivity $\sigma$ is expressed via the species concentrations as
\begin{align}
  \sigma= \frac{DF^2}{RT}\sum_{i=1}^N z_i^2 c_i
\end{align}
For a sufficiently high concentration of conducting salt, one can assume that this value
is a constant given by 
\begin{align}
  \sigma= \frac{DF^2}{RT}\sum_{i=1}^N z_i^2 c_i^{\text{bulk}}.
\end{align}

\subsection{Surface reaction terms}
We assume equality of surface and bulk chemical potentials, and that the only species on the metal
side are electrons which are given index $-1$.
Let $A_0\dots A_N$ be the reacting species. Then we can write a (surface) reaction $R_k$ as
\begin{align}
  \sum_{i=-1}^N a_{k,i} A_i
  \begin{array}{c}
     R_k^+\\
    \leftrightharpoons \\
     R_k^-\\
  \end{array}
  \sum_{i=-1}^N b_{k,i} A_i
\end{align}
Mass and charge conservation yield for $k=1\dots K$:
\begin{align}
  \sum_{i=-1}^N m_i s_{k,i}&=0\\
  \sum_{i=-1}^N z_i s_{k,i}&=0
\end{align}
where $s_{k,i}=b_{k,i}-a_{k,i}$.
Thus, with $z_{-1}=-1$ and $n=s_{k,-1}$ is the number of electrons
transferred per reaction, we have
\begin{align}
  \sum_{i=0}^N z_i s_{k,i}&=s_{k,-1}=n
\end{align}
With the affinity
\begin{align}
  \mathcal A_k = \frac{1}{RT}\sum_{i=-1}^Ns_{k,i}\mu_i
\end{align}
one writes the rate expression as
\begin{align}
  R_k(\mu_0\dots\mu_N)= R_{k,0}\left(\exp(-\beta_k\mathcal A_k) - \exp((1-\beta_k)\mathcal A_k)\right)
\end{align}
In the electroneutral case, one writes due to the equilibrium conditions in the boundary layer:
\begin{align}
  \mathcal A_k &= \frac{1}{RT}\left(\sum_{i=-1}^Ns_{k,i}\mu_i + \sum_{i=0}^Ns_{k,i}z_iF(\phi -U)\right)\\
               &= \frac{1}{RT}\left(s_{-1,i}\mu_{-1}+ \sum_{i=0}^Ns_{k,i}\mu_i + nF(\phi -U)\right)\\
               &= \frac{1}{RT}\left(\Delta g+ \sum_{i=0}^Ns_{k,i}\mu_i + nF(\phi -U)\right)\\
\end{align}

Assuming bulk ideality, we  have $\mu_i=\frac1{RT}\log c_i$, thus with $\beta=\frac12$,
\begin{align}
  R_k&= R_{k,0}\left(\exp(-\frac12\mathcal A_k) - \exp(\frac12\mathcal A_k)\right)\\
     &= R_{k,0}\left(\exp(-\frac{\Delta g}{2RT})\prod_{i=1}^nc_i^{\frac{s_i}{2}}\right)\dots
\end{align}




\section{Specific reactions}
\subsection{Ferric-ferrous}
In this section, possible models for reaction boundary
conditions  are  discussed. Their  performance for  the  three  possible
systems \eqref{sys:PNP},\eqref{sys:NNP}, and \eqref{sys:ONP} will be compared based
on numerical simulations.

For this demonstration, a further specialization of the example is performed.
Assume an aqeuous solution, $N=4$, and the species $H^+$ $(c_1)$, $Fe^{2+}$ $(c_2)$,  $Fe^{3+}$ $(c_3)$,
$SO_4^{2-}$ $(c_4)$. The redox reaction is
\begin{align}\label{eq:redox}
     Fe^{3+} + e^- &\rightleftharpoons Fe^{2+}
\end{align}
We set $R_1=0$, $R_2=R(\mu_2,\mu_3)$, $R_3=-R(\mu_2,\mu_3)$, $R_4=0$.
For  all examples, a series of stationary solutions for different
applied potentials is calculated add plotted.

We  assume  bulk  concentratiosn of  $c_1^\text{bulk}=1  mol/dm^3$  of
$H^+$,        $c_2^\text{bulk}=0.1mol/dm^3$       of        $Fe^{2+}$,
$c_2^\text{bulk}=0.1mol/dm^3$         of        $Fe^{3+}$,         and
$c_4^\text{bulk}=0.75mol/dm^3$    of   $SO_4^{2-}$.    All   diffusion
coefficients   are  equal   to  $2.0\cdot10^{-9}m^2/s$.   Further,  we
introduce two reaction constants:  $R_0=10^{-6} mol/(cm^2\cdot s)$ and
$k_0= \frac{\bar c R_0}{\sqrt{c_2^\text{bulk}c_3^\text{bulk}}}$.


Ansatz:
\begin{align}\label{eq:MBV}
  RT\mathcal A(\mu_{-1}, \mu_1,\mu_2)&=\mu_2-\mu_3-\mu_{-1}
\end{align}
Here, $\beta$ is a symmetry factor and $A$ a coefficient which can be used
``for further modeling''.
In the electroneutral case, this leads to 
\begin{align}\label{eq:MBVneu}
  R(\mu_2,\mu_3,\phi)= R_0\left(\exp\left(-\beta A\frac{\mu_2-\mu_3+zF(\phi-U)}{RT}\right)-\exp\left((1-\beta)A\frac{\mu_2-\mu_3+zF(\phi-U)}{RT}\right)\right).
\end{align}
Expressed in concentrations, one yields (setting $A=1$,$\beta=\frac12$)
\begin{align}
  R(c_2,c_3,\phi)&= R_0\left( \left(\frac{c_3}{c_2}\right)^{\frac12}\exp\left(-\frac{zF(\phi-U)}{2RT}\right)-\left(\frac{c_2}{c_3}\right)^{\frac12}\exp\left(\frac{zF(\phi-U)}{2RT}\right)\right)\\
                 &= \frac{\bar cR_0}{(c_2c_3)^{\frac12}}\left(\frac{c_3}{\bar c}\exp\left(-\frac{zF(\phi-U)}{2RT}\right)-\frac{c_2}{\bar c}\exp\left(\frac{zF(\phi-U)}{2RT}\right)\right).
\label{eq:MBVohm}
\end{align}
Replacing $\frac{\bar cR_0}{(c_2c_3)^{\frac12}}$ by $k_0=\frac{\bar cR_0}{(c_2^\text{bulk}c_3^\text{bulk})^{\frac12}}$
yields a well known form of the Butler-Volmer equation, see also the discussion below in section \ref{sec:BV}.

\subsection{Hydrogen evolution}
Species: $H^+$ ($c_1$), $H_2$ ($c_2$) $SO_4^{2-}$ ($c_3$)\\
N=2\\
Reaction:
\begin{align}
  2H^+ + 2e^- \leftrightharpoons  H_2
\end{align}

\begin{align}
  RT\mathcal A(\mu_{-1}, \mu_1,\mu_2)&=-2\mu_{-1}-2\mu_1+\mu_2\\
\end{align}



\subsection{Oxygen reduction}
Species: $H^+$ ($c_1$),  $O_2$ ($c_2$), $SO_4^{2-}$ ($c_3$)\\
N=3\\
Reaction:
\begin{align}
  4H^+ + O_2 + 4e^- \leftrightharpoons  2H_2O
\end{align}

\begin{align}
  RT \mathcal A(\mu_{-1}, \mu_0,\mu_1,\mu_2)&=-4\mu_{-1}-4\mu_1-\mu_2+2\mu_0
\end{align}



\clearpage
\bibliographystyle{unsrt}

\bibliography{lit}

\appendix
\section{Notations}

\end{document}